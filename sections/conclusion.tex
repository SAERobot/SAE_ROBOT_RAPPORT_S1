\section{Conclusion}

L'aboutissement de ce projet représente pour nous une grande première.
Certains d’entre nous ont déjà un antécédent de programmation ou d'électronique, mais cela
semble dérisoire face à tous les besoins de ce projet. Beaucoup de temps et de
travail ont été nécessaires, mais l’objectif est crucial pour la réussite de notre année. 

Même si complexe sur de nombreux points, ce projet nous a apporté beaucoup de connaissances, non seulement dans l’utilisation de logiciels comme DesignSpark PCB, TeraTerm, ARM Keil Studio, mais aussi dans le langage de programmation C++, le fonctionnement d'un microcontrôleur, ...

De plus, l'esprit de groupe est également un point important de l'apprentissage, c'est à dire savoir s'entraider, communiquer, s'organiser et planifier nos objectifs. Ce sont des points que l'on attend d'un ingénieur et qui nous seront demandés durant nos carrières professionnelles. Il est donc évident que la qualité de ce projet sera un reflet de nos capacités à évoluer au sein d'un groupe pour aller vers un même but.

Malgré les difficultés, le stress, la compétition, et parfois l'envie de casser le robot en deux, nous sommes fiers de voir notre évolution au sein de cette SAE enrichissante et vous remercions, professeurs, de nous avoir fait vivre une immersion dans le monde professionnel, le temps d'un semestre.

\noindent Merci,

\noindent L'équipe Coochie.

\vfill
\noindent\makebox[\linewidth]{\rule{.8\paperwidth}{.6pt}}\\[0.2cm]
I.U.T. Nice Côte d'Azur - SAE Robot - 2023 \hfill goofyBot
\noindent\makebox[\linewidth]{\rule{.8\paperwidth}{.6pt}}
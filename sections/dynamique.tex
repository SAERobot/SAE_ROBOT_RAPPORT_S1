\section{Dynamique du robot}
L'étude de la dynamique d'un robot à deux roues peut être effectuée en utilisant les principes de base de la mécanique des solides. La relation entre la vitesse des roues et le rayon de courbure de la trajectoire dépend de plusieurs facteurs tels que la masse du robot, la position des centres de gravité et de masse, la distribution de masse, ainsi que les propriétés des roues (rayon, frottements, stabilité, etc.).

Lorsque le robot à deux roues tourne, la vitesse angulaire de la roue interne est plus faible que celle de la roue externe, ce qui entraîne un rayon de courbure plus important pour cette dernière. Cependant, la vitesse linéaire des deux roues est la même, ce qui entraîne une différence de vitesse angulaire entre les roues. Cette différence de vitesse angulaire est proportionnelle au rayon de courbure de la trajectoire et peut être calculée en utilisant la relation suivante:

\begin{center}
    $\omega = \frac{V}{r}$
\end{center}

\noindent Avec $\omega$ la vitesse angulaire, $V$ est la vitesse linéaire et $r$ est le rayon de courbure de la trajectoire.

En conclusion, pour contrôler la dynamique d'un robot à deux roues, il est important de comprendre les relations entre les différentes grandeurs physiques telles que la vitesse des roues, le rayon de courbure de la trajectoire et la vitesse angulaire. Cela peut être utilisé pour développer des algorithmes de contrôle de robotique pour une gamme de tâches telles que le suivi de ligne.

\vfill
\noindent\makebox[\linewidth]{\rule{.8\paperwidth}{.6pt}}\\[0.2cm]
I.U.T. Nice Côte d'Azur - SAE Robot - 2023 \hfill goofyBot
\noindent\makebox[\linewidth]{\rule{.8\paperwidth}{.6pt}}
\newpage